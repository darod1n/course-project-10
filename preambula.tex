%	Преамбула

%	Шрифты
%\defaultfontfeatures{Ligatures=TeX}
%\setmainfont{Times New Roman} % Нормоконтроллеры хотят именно его
%\newfontfamily\cyrillicfont{Times New Roman}
%\setsansfont{Liberation Sans} % Тут я его не использую, но если пригодится
%\setmonofont{FreeMono} % Моноширинный шрифт для оформления кода

%	Русский язык
% Русский язык
\usepackage{polyglossia}
\setdefaultlanguage{russian}

%	Гиперссылки

%	Отступы по ГОСТ
\usepackage{geometry}
\geometry{left=3cm}
\geometry{right=1cm}
\geometry{top=2cm}
\geometry{bottom=2cm}

%	Секции без номеров(ВВЕДЕНИЕ, ЗАКЛЮЧЕНИЕ, СОКРАЩЕНИЯ ПРИНЯТЫЕ В ТЕКСТЕ....)
\newcommand{\anonsection}[1]{
	\phantomsection % Корректный переход по ссылкам в содержании
	\paragraph{\centerline{{#1}}\vspace{1.5em}}
	\addcontentsline{toc}{section}{\uppercase{#1}}
}

%	Библиография
\makeatletter
\renewenvironment{thebibliography}[1]
{\section*{\refname}
	\list{\@biblabel{\@arabic\c@enumiv}}
	{\settowidth\labelwidth{\@biblabel{#1}}
		\leftmargin\labelsep
		\itemindent 16.7mm
		\@openbib@code
		\usecounter{enumiv}
		\let\p@enumiv\@empty
		\renewcommand\theenumiv{\@arabic\c@enumiv}
	}
	\setlength{\itemsep}{0pt}
}
\makeatother