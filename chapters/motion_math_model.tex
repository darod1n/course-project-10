%	Математическая модель движения ЦМ ВА
\section{Математическая модель движения ЦМ ВА}

Математическая модель пространственного движения аэродинамического летательного аппарата (ЛА) представим в виде системы дифференциальных уравнений (ДУ). ЛА как незакрепленное тело имеет шесть степеней свободы (три - поступательного движения, три - углового (вращательного)). Выбор переменных, описывающих изменение по этим степеням свободы, предопределен использованием вполне определенного набора систем координат (СК), а именно используются ИГГСК(инерциальная геоцентрическая гироскопическая СК) $O\xi \eta \zeta$ и ССК(связанная с ЛА СК) $O x_1 y_1 z_1$. Необходимо описать изменение выбранных координат в зависимости от скоростей (кинематика движения), а также изменение скоростей под действием влияющих факторов - сил и моментов (динамика движения).

Уравнения динамики поступательного движения для различных целей удобно составлять в проекциях на оси траекторной или связанной СК.

\clearpage

\subsection{Посадочная система координат}
\clearpage

\subsection{Связанная система координат}
\clearpage

\subsection{Уравнения динамики движения ЦМ в траекторной СК}
Скорость движения ЦМ в траекторной системе $\overline{V}_k = (V_k, \  0, \  0)^T$. Переносная угловая скорость $\overline{\Omega}$ в рассматриваемом случае - это скорость вращения траекторной СК относительно нормальной. Обозначим проекции этой угловой скорости на оси траекторной системы $\Omega_{xk}, \Omega_{yk}, \Omega_{zk}$, т.е. $\overline{\Omega}_k = (\Omega_{xk}, \Omega_{yk}, \Omega_{zk})^T$.

Угловое положение траекторной системы относительно нормальной описывается углами пути $\Psi$ и наклона траектории $\theta$, а изменение этого положения (вращение) - соответствующими скоростями $\Dot{\overline{\Psi}}$ и $\Dot{\overline{\theta}}$, причем вектор $\Dot{\overline{\Psi}}$ направлен вдоль вертикальной оси $y_g$, а вектор $\Dot{\overline{\theta}}$ - вдоль оси $z_k$ траекторной системы. Спроецировав эти вектора на оси траекторной системы, можно  получить нужные проекции переносной скорости $\Omega_{xk} = \Dot{\Psi} \sin{\theta}$, $\Omega_{yk} = \Dot{\Psi} \cos{\theta}$, $\Omega_{zk} =\Dot{\theta}$.

В результате получаем:
\begin{equation}
m 
\begin{bmatrix}
\Dot{V}_k \\
0\\
0
\end{bmatrix} + m
\begin{bmatrix}
\Dot{\Psi} \sin{\theta} \\
\Dot{\Psi} \cos{\theta}\\
0
\end{bmatrix} \times 
\begin{bmatrix} V_k\\
0\\
0\\
\end{bmatrix} = \overline{F}_k
\end{equation}
после выполнения векторного произведения
\begin{equation}
m 
\begin{bmatrix}
\Dot{V}_k \\
0\\
0
\end{bmatrix} + m
\begin{bmatrix} 0\\
V_k \Dot{\theta}\\
- V_k \Dot{\Psi} \cos{\theta}\\
\end{bmatrix} = \overline{F}_k
\end{equation}
или в более компактном виде 
\begin{equation}
m 
\begin{bmatrix}
\Dot{V}_k \\
V_k \Dot{\theta}\\
- V_k \Dot{\Psi} \cos{\theta}
\end{bmatrix} = \overline{F}_k
\label{eq:ur_1k}
\end{equation}
где $\overline{F}_k$ - вектор результирующий всех внешних сил, представленный своими проекциями на оси траекторной системы.

Векторное уравнение ~(\ref{eq:ur_1k}) соответствует системе из трех скалярных уравнений, которые после приведения к нормальной форме при $\theta \neq \pm 90^{\circ}$ приобретут вид

\begin{equation}
\dot{V}_k = \frac{F_{xk}}{m},
\end{equation}
\begin{equation}
\dot{\theta} = \frac{F_{yk}}{mV_k},
\end{equation}
\begin{equation}
\dot{\psi} =- \frac{F_{zk}}{mV_k\cos{\theta}}
\end{equation}
\clearpage
\subsection{Уравнения динамики движения ЦМ в связанной СК}

Переносной угловой скоростью в этом случае является вектор угловой скорости вращения связанной СК относительно нормальной $\overline{\omega} = (\omega_x, \omega_y, \omega_z)^T$, проекциями которого на оси связанной системы являются угловые скорости крена $\omega_x$, рыскания $\omega_y$ и тангажа $\omega_z$. Следует обратить внимание на то, что эти угловые скорости являются теми величинами, которые непосредственно могут быть измерены на борту ЛА.

Обозначив проекции вектора земной скорости на оси связанной СК $V_{kx}$, $V_{ky}$ и $V_{kz}$ соответственно, т. е. записав $\overline{V}_{k\text{св}} = (V_{kx}, V_{ky}, V_{kz})^T$ и подставив вектора $\overline{\omega}$ и $\overline{V}_{k\text{св}}$ в уравнение ~(\ref{eq:eq_control})
\clearpage

\subsection{Математическая модель ориентации}
\clearpage