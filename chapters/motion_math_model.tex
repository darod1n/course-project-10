%	Математическая модель движения ЦМ ВА
\section{Математическая модель движения ЦМ ВА}
Модель движения ЦМ запишем в следующем виде:
\begin{equation}
	\left\{ \begin{matrix}
			\dot{W}_{X1} = \frac{F_{X1}}{m} \\
			\dot{W}_{Y1} = \frac{F_{Y1}}{m} \\
			\dot{W}_{Z1} = \frac{F_{Z1}}{m} \\
		\end{matrix} \right. ,
\end{equation}
где $\dot{W}_{X1}$, $\dot{W}_{Y1}$, $\dot{W}_{z1}$ - проекции кажущегося ускорения на оси связанной системы координат, $F_{X1}$, $F_{Y1}$, $F_{Z1}$ - проекции сил действующих на ВА, $m$ - масса ВА.

\begin{equation}
	\begin{gathered}
		F_{X1} = X - R_{\sum}, \\ 
		F_{Y1} = F_{\text{упр}Y1} + Y, \\
		F_{Z1} = F_{\text{упр}Z1} + Z,
	\end{gathered}
\end{equation}
где $R_{\sum}$ - суммарная сила тяги действующая по оси $OX_1$ связанной системы координат, $F_{\text{упр}Y1}$ - управляющая сила действующая по оси $OY_1$ связанной системы координат, $F_{\text{упр}Z1}$ - управляющая сила действующая по оси $OZ_1$ связанной системы координат, $X$ - продольная составляющая аэродинамической силы, $Y$ - нормальная составляющая аэродинамической силы, $Z$ - поперечная составляющая аэродинамической силы.
\clearpage

\subsection{Расчет управляющих сил}

На ЛА во время полета также действуют управляющие силы и моменты, которые создаются с помощью СУБ

%рисунок направление тяги
Угол наклона сопел равен $\alpha_c = 5.73^{\circ}$ показан на рисунке , максимальная суммарная тяга равна $$R_{max}^{\sum} = 22 \text{Тс}$$

Максимальная тяга $i$ - го сопла $$R_{\max i} = 2.75 \text{Тс}$$

Минимальная суммарная тяга $$R^{\sum}_{\min} = 8 \text{Тс}$$

Минимальная тяга $i$ - го сопла $$R_{\min i} = 1 \text{Тс}$$

\clearpage

\subsection{Расчет аэродинамических сил}
При полете ЛА в атмосфере на них действует сопротивление воздуха, называемое аэродинамическим.

Аэродинамическая сила $R_A$ складывается из сил давления воздуха, направленных по нормалям к поверхности ЛА, и сил трения воздуха касательных к ней. 

Для определения $R_{A}$ используют формулу:
\begin{equation}
	R_{A} = C_{R} \cdot q \cdot S, 
\end{equation}
где $q = \frac{\rho \cdot V^2}{2}$ - скоростной напор набегающего невозмущенного потока, $V$ - скорость набегающего потока, $\rho$ - плотность атмосферы, $S_\text{м}$ - характерная площадь ракеты, $M$ - число маха ракеты, $C_R$ - безразмерный аэродинамический коэффициент, зависящий от формы ракеты.

Запишем формулу для нахождения аэродинамической силы в плоскости $O Y_1 Z_1$:
\begin{equation}
	R_{Y_1 Z_1} = C_{Y1} (\alpha^*, M) \cdot q \cdot S_M
\end{equation}

Продольную составляющую аэродинамической силы запишем следующим образом: 
\begin{equation}
	X = C_{X_1} \cdot q \cdot S_M
\end{equation}

Нормальная составляющая аэродинамической силы:
\begin{equation}
	Y = R_{Y_1 Z_1} \cdot cos \varphi
\end{equation}

Поперечная составляющая аэродинамической силы:
\begin{equation}
	Z = -R_{Y_1 Z_1} sin \varphi
\end{equation}



\clearpage