%	Общая постановка задачи синтеза СУ ПТДУ
\section{Общая постановка задачи синтеза СУ ПТДУ}
Исходя из требований ТЗ, при н. у. которые были определены ранее решенной задачей определения момента включения ПТДУ, требуется переместить объект из одной из возможных точек области н. у. в заданную точку пространства (точка зависания) за заданное время. В связи с тем, что органами управления являются не поворачиваемые сопла, т. о. сама задача определяется движением ЦМ и угловым движением. В связи с компоновкой сопловых блоков ВА эти задачи взаимосвязаны. Для того чтобы упростить задачу на данном этапе синтеза будем считать что есть приведенные моменты и приведенные силы. Распределения сил и моментов по соплам это задача следующего этапа. Чтобы синтез управления упростить мы декомпозируем систему на две независимые подсистемы: управления ЦМ и Управления угловым положением на основе разделения движений. Считаем, что динамика требуемой угловой ориентации на порядок быстрее чем движения ЦМ.

При этом каждой из этих подсистем управления выдвигаются свои требования по качеству:
\begin{enumerate}
	\item Время переходного процесса
	\item Осутствие перерегулирования
	\item Точность вывода в точку при вариации н. у. в области, которая ранее была определена алгоритмом включения ПТДУ
\end{enumerate}
\newpage

\subsection{Постановка задачи управления ЦМ}

Цель управления: 
\begin{equation}
|| x(t) - x_{*}(t) || \leq \Delta_x, \  \forall t \geq \hat{t}
\label{eq:ur_target_control}
\end{equation}
где $x(t) = (\xi \ \dot{\xi} \ \eta\  \dot{\eta} \ \zeta \ \dot{\zeta})^T$ - вектор состояния объекта управления относительно центра масс, $x_{*}(t)$ - желаемая траектория спускаемого аппарата с заданными показателями качества.

Это была общая постановка задачи управления. Для того чтобы решить данную задачу мне нужно прежде всего сформировать желаемую траекторию и стабилизироваться относительно 
её.

Таким образом, задача распадается на две подзадачи:
\begin{enumerate}
	\item Синтез желаемой кинематической траектории с учетом желаемой динамики движения по данной траектории.
	\item Стабилизация ЦМ ВА относительно желаемой траектории (задача слежения)
\end{enumerate}
\clearpage