%	Введение
\anonsection{Введение}

Целью данного курсового проекта является синтез алгоритма наведения и стабилизации центра масс возвращаемого аппарата. Разработка алгоритмов и программного обеспечения математического моделирования замкнутой системы управления в MatLab. Проведение сравнительного анализа результатов моделирования.


Под синтезом системы автоматического управления понимается направленный расчет, имеющий конечной целью отыскание рациональной структуры системы и установление оптимальных величин параметров ее отдельных звеньев. 

Синтез можно трактовать как инженерную задачу, сводящуюся к такому построению системы, при котором обеспечивается выполнение технических требований к ней. Подразумевается, что из многих возможных решений инженер, проектирующий систему, будет выбирать те, которые являются оптимальными с точки зрения существующих конкретных условий и требований к габаритам, весу, простоте, надежности и т. п.

При инженерном синтезе системы автоматического управления необходимо обеспечить, во-первых, требуемую точность и, во-вторых, приемлемый характер переходных процессов.

Обеспечение приемлемых переходных процессов оказывается почти всегда более трудным вследствие большого числа варьируемых параметров и многозначности решения задачи демпфирования системы. Поэтому существующие инженерные методы часто ограничиваются решением только второй задачи, так как обеспечение требуемой точности может быть достаточно просто сделано на основании использования существующих критериев точности и совершенствования их практически не требуется.

В настоящее время для целей синтеза систем автоматического управления широко используются вычислительные машины, позволяющие производить полное или частичное моделирование проектируемой системы
\clearpage