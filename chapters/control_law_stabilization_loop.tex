\section{Законы управления в контуре стабилизации}
Для управления двигателем вычисляется командный сигнал, состоящий из четырех составляющих 
\begin{equation}
	\begin{gathered}
		\delta_R = \delta^0 + K_0 \cdot \left( W_c (j \lambda) \cdot P_{cp} - \overline{P}_\text{пр} (\overline{R}_\text{ТР}) \right) + K_1 \cdot \left( \dot{W}_{X_1} - \dot{W}_\text{ТР} \right) \\ + K_2 \cdot \left( V_{\eta} - V_{\eta_\text{ТР}} \right) 
	\end{gathered}
\end{equation}
$P_{\text{ср}}$ - величина давления в камере сгорания. Увеличению командного сигнала соответствует открытие СБ.

Первая составляющая $\delta^0$ - уставка, определяющая режим работы ПТДУ, задается априорно и интерполируется в функции требуемой величины тяги по ходу процесса.

Вторая составляющая контролирует процесс в камере сгорания; $W_c (j \lambda)$ - интегро-дифференцирующий цифровой фильтр, стабилизирующий процесс в КС.

Третья составляющая следит за выполнением основной задачи - созданием требуемого кажущегося ускорения по продольной оси изделия.

Четвертая составляющая используется на финишном участке при реализации постоянной вертикальной скорости.

Интегральная составляющая в законе управления отсутствует ввиду переменного характера требуемой величины тяги. Требуемая точность регулирования достигается достаточным коэффициентом усиления.

Командный сигнал уставки ограничивается по величине отрицательного градиента
\begin{equation}
	\begin{gathered}
		\overline{P}_\text{пр} (nT_c) = \overline{P}_\text{пр} \left[ (n-1)T_c \right] + \Delta P_\text{пр} \\
		\Delta P_\text{пр} = P_\text{пр} (R_\text{ТР}) - \overline{P}_\text{пр} (R_\text{ТР})
	\end{gathered}
\end{equation}

если $\dot{P}_\text{пр}<0$ и $|\Delta P| > \Delta P(P_\text{пр})$
\begin{equation}
	\Delta P = - \Delta P (P_\text{пр})
\end{equation}

если $\dot{P}_\text{пр} \geq 0 $, то $\Delta P => \Delta P^{\max}$
\begin{equation}
	\dot{P}_\text{пр} = P_\text{пр}(nT_c) - P_\text{пр}[(n-1)T_c]
\end{equation}
$\Delta P (P_\text{пр})$ - интерполируется как монотонно возрастающая функция при возрастании $P_\text{пр}$; $T_c$ - цикл решения задачи стабилизации давления, $T_c = 0.25\cdot T_0$; $T_{0c}$ - цикл выдачи сигналов ССД на органы управления, $T_{0c} = 0.5 \cdot T_0$.

На спад давления обеспечивается ограничение градиента спада тяги на уровне $3000 \frac{кГс}{с}$

Командный сигнал $\delta_R$ ограничивается с учетом допустимого отклонения органов управления
\begin{equation}
	\overline{\delta})_R = \left\{ \begin{matrix} \delta_R, \ \  \text{если} \  |\delta_R| < \delta_{\text{ОГР}} \\ \delta_\text{ОГР} \cdot sign \delta_R, \text{если} \ |\delta_R | \geq \delta_\text{ОГР} \end{matrix} \right.
\end{equation}

\clearpage
