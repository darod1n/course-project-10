\section{Системы координат}
В работе используются посадочная система координат и связанная с ВА система координат.

Задачи наведения решаются в посадочной системе координат (ПСК):
\begin{itemize}
	\item начало координат - в точке посадки;
	\item Ось $O \eta$ - по местной вертикали;
	\item Ось $O \xi$ - по касательной к меридиану в точке посадки и направлена на север;
	\item Ось $O \zeta$ - дополняет систему до правой;
\end{itemize}


Силы и моменты, действующие на ЛА рассмотрим в связанной системе координат $O X_1 Y_1 Z_1$ - правая прямоугольная система координат, связанная с изделием (ССК):
\begin{itemize}
	\item Ось $O X_1$ - совпадает с продольной осью симметрии изделия и направлена в сторону хвостовой части изделия;
	\item Ось $O Y_1$ - направлена в сторону I строительной плоскости изделия;
	\item Ось $O Z_1$ - дополняет систему координат до правой;
\end{itemize}

\clearpage

\subsection{Посадочная система координат}


\clearpage

\subsection{Связанная система координат}

\clearpage

\subsection{Матрица перехода из ПСК в ССК}
Для составления матрицы перехода из ПСК в ССК необходимо составить элементарные матрицы последовательного поворота на один угол и перемножить их. Повороты будем осуществлять против часовой стрелки, т. е. в положительном направлении, в следующей последовательности: $\vartheta \rightarrow \psi \rightarrow \varphi \rightarrow$ как показано на рис. 
% Вставить рисунок взаимного располождения 

Для первого поворота на угол $\vartheta$, получим: 
\begin{equation}
	A_\vartheta = \begin{bmatrix}
		\cos \vartheta & \sin \vartheta & 0 \\
		- \sin\vartheta & \cos \vartheta & 0 \\
		0 &	& 1 
	\end{bmatrix},
\end{equation}
для второго поворота на угол $\psi$
\begin{equation}
	A_\psi = \begin{bmatrix}
		\cos \psi & 0 & -\sin \psi  \\
		0 & 1 & 0 \\
		-\sin \psi & 0 & \cos \psi
	\end{bmatrix},
\end{equation}
для третьего поворота на угол $\phi$
\begin{equation}
	A_\varphi = \begin{bmatrix}
		1 & 0 & 0  \\
		0 & \cos \varphi & \sin \varphi \\
		0 & -\sin \varphi  & \varphi
	\end{bmatrix}
\end{equation}

Произведение трех матриц имеет вид:
\begin{equation}
	A = A_\varphi A_\psi A_\vartheta
\end{equation}

В результате перемножения получим матрицу перехода от ПСК к ССК: 
\begin{equation}
	A = \begin{bmatrix}
		a_{11} & a_{12} & a_{13} \\
		a_{21} & a_{22} & a_{23} \\
		a_{31} & a_{32} & a_{23} 
	\end{bmatrix},
\end{equation}
где
\begin{equation}
	\begin{gathered}
		a_{11} = \cos \vartheta \cdot \cos \varphi \\
		a_{12} = \cos \vartheta \cdot \sin \varphi \cdot \sin \psi - \sin \vartheta \cdot \cos \varphi \\
		a_{13} = \cos \vartheta \cdot \sin \psi \cdot \cos \varphi + \sin \vartheta \cdot \sin \varphi \\
		a_{21} = \sin \vartheta \cdot \cos \psi \\
		a_{22} = \sin \vartheta \cdot \sin \varphi \cdot \sin \psi + \cos \vartheta \cdot \cos \varphi \\
		a_{23} = \sin \vartheta \cdot \sin \psi \cdot \cos \varphi - \cos \vartheta \cdot \sin \varphi \\
		a_{31} = - \sin \psi \\
		a_{31} = \cos \psi \cdot \sin \varphi \\
		a_{31} = \cos \psi \cdot \cos \varphi \\
	\end{gathered}
\end{equation}

Формула перехода имеет вид:
\begin{equation}
	\begin{bmatrix}
		\xi \\
		\eta \\
		\zeta
	\end{bmatrix} = A \cdot \begin{bmatrix}	X_1 \\ Y_1 \\ Z_1 \end{bmatrix}
\end{equation}
\clearpage