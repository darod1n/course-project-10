%	Синтез желаемой кинематической траектории
\section{Синтез желаемой кинематической траектории}

Мы имеем начальные условия и конечные координаты траектории на момент времени T. Кинематическая функция должна быть гладкой функцией хотя бы второго порядка.

Будем искать решение кинематической траектории в виде многочлена на единице меньше, чем количество краевых условий. Это связано с тем, что у многочлена степени $n$ в общем случае имеется $(n + 1)$ параметров, которые подлежат выбору. 

Краевыми условиями являются начальные и конечные координаты, скорости и ускорения. Т.о. в качестве функций описывающих траектории движения выберем полиномы пятого порядка:
\begin{equation}
	\alpha(t) = \alpha_0 + \alpha_1 t + \alpha_2 t^2 + \alpha_3 t^3 + \alpha_3 t^4 + \alpha_5 t^5 
	\label{eq:ur_traectorii}
\end{equation}

\begin{equation}
	\Dot{\alpha}(t) = \alpha_1 + \alpha_2 t + \alpha_3 t^2 + \alpha_3 t^3 + \alpha_5 t^4 
	\label{eq:ur_velocity}
\end{equation}

\begin{equation}
	\Ddot{\alpha}(t) = 2 \alpha_2 + 6 \alpha_3 t + 12 \alpha_4 t^2 + 20 \alpha_5 t^3
\end{equation}
Где $\alpha(t)$ любая из координат $\xi(t)$, $\eta(t)$, $\zeta(t)$, $t$ - время, на котором действуют коэффициенты $a_0-a_5$.
\clearpage

\subsection{Алгоритмы наведения}
\clearpage

\subsection{Результаты моделирования}
\clearpage