%	Синтез желаемой кинематической траектории
\section{Алгоритм наведения}

Мы имеем начальные условия и конечные координаты траектории на момент времени T. Кинематическая функция должна быть гладкой функцией хотя бы второго порядка.

Будем искать решение кинематической траектории в виде многочлена на единице меньше, чем количество краевых условий. Это связано с тем, что у многочлена степени $n$ в общем случае имеется $(n + 1)$ параметров, которые подлежат выбору. 

Краевыми условиями являются начальные и конечные координаты, скорости и ускорения. Т.о. в качестве функций описывающих траектории движения выберем полиномы пятого порядка:
\begin{equation}
	\alpha(t) = \alpha_0 + \alpha_1 t + \alpha_2 t^2 + \alpha_3 t^3 + \alpha_3 t^4 + \alpha_5 t^5 
	\label{eq:ur_traectorii}
\end{equation}

\begin{equation}
	\Dot{\alpha}(t) = \alpha_1 + \alpha_2 t + \alpha_3 t^2 + \alpha_3 t^3 + \alpha_5 t^4 
	\label{eq:ur_velocity}
\end{equation}

\begin{equation}
	\Ddot{\alpha}(t) = 2 \alpha_2 + 6 \alpha_3 t + 12 \alpha_4 t^2 + 20 \alpha_5 t^3
\end{equation}
Где $\alpha(t)$ любая из координат $\xi(t)$, $\eta(t)$, $\zeta(t)$, $t$ - время, на котором действуют коэффициенты $a_0-a_5$. 

Задаваясь начальными и конечными значениями координат $\alpha (0)$, $\dot{\alpha} (0)$, $\ddot{\alpha} (0)$, и $\alpha (T)$, $\dot{\alpha} (T)$, $\ddot{\alpha} (T)$ получаем три тройки уравнений относительно $a_{3 \xi}$, $a_{3 \eta}$ , $a_{3 \zeta}$, $a_{4 \xi}$, $a_{4 \eta}$ , $a_{4 \zeta}$, $a_{5 \xi}$, $a_{5 \eta}$ , $a_{5 \zeta}$ добавляя к ним коэффициенты $a_{0 \xi}$, $a_{0 \eta}$ , $a_{0 \zeta}$, $a_{1 \xi}$, $a_{1 \eta}$ , $a_{1 \zeta}$, $a_{2 \xi}$, $a_{2 \eta}$ , $a_{2 \zeta}$, которые определяются начальными условиями, имеем следующее решение для определения многочленов пятого порядка:
\begin{equation}
	\begin{gathered}
		a_0 = \alpha (0), \\
		a_1 = \dot{\alpha} (0) \\
		a_2 = \ddot{\alpha} (0) \\
		a_3 = \frac{10 \cdot ( a(T) - a(0) ) }{T^3} - \frac{4 \cdot \dot{a}(T) + 6 * \dot{a}(0) ) }{T^2} +  \frac{ \ddot{a}(T) - 3 \cdot \ddot{a}(0) }{2 \cdot T}\\
		a_4 = \frac{-15 \cdot ( a(T) - a(0) ) }{T^4} + \frac{7 \cdot \dot{a}(T) - 8 * \dot{a}(0) ) }{T^3} +  \frac{ 1.5 \cdot \ddot{a}(T) -  \ddot{a}(0) }{T^3} \\
		a_5 = \frac{6 \cdot ( a(T) - a(0) ) }{T^5} + \frac{3 \cdot \dot{a}(T) - \dot{a}(0) ) }{T^4} +  \frac{\ddot{a}(T) -\ddot{a}(0) }{T^3} \\
	\end{gathered}
\end{equation}
$T$ - время, оставшееся до окончания процесса, начальное значение которого определяется в момента запуска ПТДУ.
\clearpage

\subsection{Формирование вектора требуемой ориентации ВА}
Требуемая ориентация ветора тяги находится из компонент вектора требуемого кажущегося ускорения в виде соответствующих направляющих косинусов
\begin{equation}
	\overline{e}^R = (e^{R}_\xi, e^{R}_\eta, e^{R}_\zeta)
\end{equation}

\begin{equation}
	\begin{gathered}
		\dot{W}^\text{ТР}_\xi = \dot{V}_\xi, \\
		\dot{W}^\text{ТР}_\eta = \dot{V}_\eta + g, \\
		\dot{W}^\text{ТР}_\zeta = \dot{V}_\zeta,
	\end{gathered}
\end{equation}

Рассчитываем требуемое кажущиеся ускорение ЛА
\begin{equation}
	\dot{W}_\text{ТР} = \sqrt{ \dot{W}^{\text{ТР}^2}_\xi + \dot{W}^{\text{ТР}^2}_\eta +\dot{W}^{\text{ТР}^2}_\zeta }
\end{equation}

Так как требуемое направление оси $X$ противоположно направлению тяги, выражения принимают следующий вид:
\begin{equation}
	\begin{gathered}
		e^R_\xi = - \frac{\dot{W}^\text{ТР}_\xi}{\dot{W}_\text{ТР}}, \\
		e^R_\eta = - \frac{\dot{W}^\text{ТР}_\eta}{\dot{W}_\text{ТР}}, \\
		e^R_\zeta = - \frac{\dot{W}^\text{ТР}_\zeta}{\dot{W}_\text{ТР}} \\
	\end{gathered}
\end{equation}
\clearpage
