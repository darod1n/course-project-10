\section{Схема действия}
Перед запуском ПТДУ проводится тестирование тракта исполнительных механизмов (РМ), производится установка стартовых величин эффективных сечений сопловых блоков (СБ). По завершению тестового режима подается сигал на зажигание ПТДУ. В командных сигналах на РМ СБ учитываются демпфирующие составляющие закона управления угловым положением аппарата. Таким образом, уже при запуске ДУ предотвращается развитие неустойчивого процесса. Это участок успокоения - т. е. гашение угловых скоростей и разворот аппарата продольной осью вдоль вектора требуемой ориентации, вырабатываемого терминальным регулятором. По крену аппарат разворачивается таким образом, что вектор $\overline{Z}_1$ аппарата ориентируется ортогонально вектору $\overline{\xi}$ ПСК в положительном направлении $\overline{\zeta}$. Затем по оперативно рассчитываемой программе аппарат затормаживается и выводится в точку посадки на высоту ~ $20 \text{м}$ с вертикальной скоростю $-2 \div -3 \frac{\text{м}}{\text{c}}$. Управление на этом участке заключается в поддержании постоянной вертикальной скорости и стабилизации аппарата в горизонтальной плоскости.

\clearpage