%	Анализ технической реализуемости
\section{Анализ технической реализуемости системы стабилизации ЦМ}

Для реализации вышеописанной системы управления нам понадобятся такие технические средства, которые позволили бы нам измерять положение, скорость, ускорение нашего аппарата в пространстве.
\clearpage

\subsection{Выбор датчиков}

Акселерометры предназначаются для измерения ускорений движущихся объектов и для преобразования этих ускорений в сигнал, используемый для определения параметров траектории  движения объекта или для целей автоматического управления этой траекторией. Акселерометры применяются для измерения линейных и угловых ускорений. В соответствии с этим они называются линейными акселерометрами или угловыми акселерометрами.

По назначению различают следующие акселерометры: для визуального контроля, для систем телеметрического контроля, для систем инерциальной навигации, для систем автоматического управления.

По исполнению акселерометры подразделяются на следующие две группы:\begin{itemize}
	\item пружинные, построенные по разомкнутой структурной схеме;
	\item компенсационные, построенные по замкнутой структурной схеме.
\end{itemize}

Компенсационные акселерометры, в свою очередь, делятся на акселерометры с позиционной обратной связью (акселерометры с «электрической пружиной»), со скоростной обратной связью (интегрирующие акселерометры) и с обратной связью по ускорению (акселерометры с двойным интегрированием). Акселерометры выполняют с непрерывным выходным сигналом или с дискретным.

Наиболее широкое применение акселерометры получили на летательных аппаратах. Как линейное, так и угловое ускорение движущегося в пространстве летательного аппарата можно в каждый момент времени разложить на три составляющие в системе координат, связанной с летательным аппаратом и ориентированной по его главным осям (осям симметрии).

Для получения полной информации о линейных и угловых ускорениях летательного аппарата необходимо иметь шесть акселерометров (три линейных и три угловых), измерительные оси которых ориентированы по главным осям летательного аппарата и каждый из которых измеряет соответствующий компонент линейного или углового ускорения.

В системах автоматического управления траекторией полета иногда используют не полную информацию, а лишь некоторую ее часть, например ограничиваются применением двух линейных акселерометров, измеряющих компоненты линейных ускорений по поперечным осям летательного аппарата.

При использовании акселерометров в системах инерциальной навигации применяют два линейных акселерометра, измерительные оси которых ориентированы по двум взаимно перпендикулярным направлениям, лежащим в горизонтальной плоскости, причем одно из направлений обычно совмещают с плоскостью географического меридиана. Возможны и другие способы ориентации измерительных осей акселерометров в зависимости от выбранной системы координат.
\clearpage

\subsection{Реализация на БЦВК}

Бортовой цифровой вычислительный комплекс получает информацию от датчиков и бортовых систем, обрабатывает ее в режиме разделения времени между задачами и выдает управляющие воздействия на исполнительные органы и бортовые системы. Для обеспечения работы в реальном масштабе времени каждые 32,8 мс прерывается работа процессора, что создает предпосылки для периодического возвращения к решению одних и тех же задач.

Облик и структуру БЦВК во многом определяет требование сохранения работоспособности и обеспечения безопасности экипажа при любых двух отказах. В состав БЦВК входят две идентичные по структуре и оборудованию вычислительные системы: центральная (ЦВС) и периферийная (ПВС), каждая из которых включает в себя четыре бортовые цифровые вычислительные машины, работающие синхронно по одинаковым программам, фактически резервирующие друг друга и представляющие четыре параллельных канала, на выходе каждого из которых имеется встроенная резервированная схема сравнения, контролирующая команды, выдаваемые абонентам из всех четырех БЦВМ. При отказе одной из БЦВМ схема сравнения блокирует ее выход и вычислительная система продолжает работать в составе трех каналов, при отказе второй БЦВМ ситуация повторяется: выход отказавшей БЦВМ блокируется и система продолжает работать в составе двух каналов.

Как известно, программная синхронизация четырех БЦВМ в реальном масштабе времени при любом сочетании допустимых отказов является чрезвычайно сложной и недостаточно надежной. В связи с этим в ЦВС и ПВС используется не программная, а аппаратная синхронизация, для чего в составе БЦВК имеется единый кварцевый генератор, подающий во все восемь БЦВМ единую сетку тактовых импульсов частотой 4 МГц и с периодом прерывания 32,8 мс. Поскольку задающий генератор также должен удовлетворять требованию "надежная работа при двух отказах", он имеет пять каналов резервирования, на выходе каждого из которых установлена схема голосования "три из пяти". Кроме того, в состав БЦВК входит накопитель на магнитной ленте (МЛ) емкостью 819200 32-разрядных слов для хранения программного обеспечения и загрузки его в оперативную память БЦВК в процессе полета.

В связи с тем, что нам нужно хранить и обрабатывать данные предполагается, что это будет происходить в бортовом цифровом вычислительном комплексе, при этом надо учитывать шаг дискретизации синтезируемых алгоритмов управления которые влияют на точность. Бортовая цифровая вычислительная техника — оборудование, входящее в единый комплекс и предназначенное для обеспечения сбора и обработки данных. В процессе работы бортовой цифровой вычислительный комплекс после сбора данных со всех систем и последующей обработки выдает управляющие воздействия на бортовые системы и исполнительные органы управления. Чтобы процесс шел в реальном времени, через определенные промежутки времени необходимо прерывать работу процессора для периодического возращения процессора к решению одних и тех же рабочих задач. Отличие БЦВМ от различных специализированных вычислителей и блоков обработки данных (которых в современном самолёте предостаточно) в том, что БЦВМ имеют общепринятую для компьютеров структуру: наличие оперативной и долговременной памяти, устройств ввода-вывода и т. д. Важной особенностью управления бортовыми системами является программное управление их резервированием. Сложная логика управления избыточностью требует проведения коммутации соответствующих схем и элементов строго по циклограммам управления, поэтому БЦВК не только анализирует числовые значения контрольных величин, но и задает и контролирует временные соотношения в ходе выполнения полетных задач. Предполагаем, что эта задача будет рассмотрена на следующем этапе разработки.
\clearpage