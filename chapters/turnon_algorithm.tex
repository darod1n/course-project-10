%	Алгоритм определения момента включения ПТДУ
\section{Алгоритм определения момента включения ПТДУ}
Исходя из независимости движений в ортогональных направлениях и приоритете вертикального канала в процессе управления, выбор момента включения ДУ определяется высотой, скоростью движения аппарата и необходимостью обеспечения наиболее предпочтительного режима работы ПТДУ.

Предпочтительным режимом является работа в середине линейного диапазона регулирования тяги, так как при действии возмущений такой режим создает наилучшие условия для управления.

В связи с тем, что аппарат на момент включения ПТДУ движется практически с установившейся скоростью, суммарная потенциальная и кинетическая энергия, которую необходимо рассеять с помощью ПТДУ, убывает при снижении аппарата:

\begin{equation}
E(h) = \frac{mv^2}{2} + mgh
\end{equation}
где $v \ \approx const$

Принимая равенство совершаемой двигателем работы и энергии аппарата, погашаемой на участке спуска при величине тяги двигателя, соответствующей середине диапазона регулирования $E = R_{cp} \eta_0$, получаем высоту включения ПТДУ:

\begin{equation}
\eta_0 = \frac{m v^2}{2} \cdot \frac{1}{R_{cp} + C_x \rho S_m \frac{v^2_\eta}{6} - mg}
\end{equation}
здесь $R_{cp} = 12$Тс.

Этот функционал может вычисляться на борту и при возможном разбросе начальной скорости $v_{\eta_0}$ в пределах $80 \div 120 \frac{\text{м}}{\text{с}}$, высота включения ПТДУ лежит в пределах $\eta_0 = 400 \div 800\text{м}$. 
Предполагая, что в вертикальном канале реализуется движение с постоянным замедлением, т. е. $\Dot{V}_\eta = const$, получаем интегралы движения:
\begin{equation}
\eta = \eta_0 - V_\eta \cdot t + \frac{\Dot{V}_\eta \cdot t^2}{2}
\label{eq:int-eq}
\end{equation}
\begin{equation}
V_\eta = -V_{\eta_0} + \Dot{V}_\eta \cdot t
\end{equation}
откуда
\begin{equation}
\Dot{V}_\eta = \frac{V^2_{\eta_0}}{2 \eta_0}
\label{eq:usk_v_eta}
\end{equation}
\begin{equation}
T = -\frac{V_{\eta_0}}{\Dot{V}_\eta}
\end{equation}

Среднее значение вертикального ускорения, обеспечивающего затормаживание на заданной высоте $\Dot{v}_\eta \approx 12 \div 20 \frac{\text{м}}{\text{с}^2}$, и время торможения $T \approx 8 -10 $с. При этом минимальный суммарный импульс тяги, расходуемый на затормаживание аппарата равен $\approx 150 \text{Тс} \cdot \text{с}$. На участке вертикального спуска расходуется еще $\approx 70 \div 80 \text{Тс} \cdot \text{с}$.

Минимальная высота включения ПТДУ, при которой еще возможно
погасить энергию аппарата $h_0 \approx 400 \text{м}$, время торможения $\approx 7 \text{c}$, тяга $R_{max} = 22 \text{Тс}$. При этом перемещения в горизонтальной плоскости исключаются. Весь импульс при этом сбрасывается в вертикальном канале управления.
\clearpage